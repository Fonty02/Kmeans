\section{Implementazioni}
\subsection{Progetto Base}
\noindent La versione base del progetto consiste in un'architettura client/server che permette all'utente lo studio di cluster. Il server dovrà essere eseguito su una macchina con un database MySQL in esecuzione. 
Il servizio sarà raggiungibile sulla porta 8080, e potrà comunicare con diversi client contemporaneamente. 
\\ I servizi offerti dal server all'utente tramite il client CLI sono i seguenti:
\begin{itemize}[label=-]
  \item \textbf{scoperta di cluster} fornendo al server il nome della tabella presente nel database ed il numero di cluster da scoprire
  \item \textbf{salvataggio dei cluster} generati nella macchina dove il server viene eseguito. Il salvataggio avverrà in automatico alla generazione di nuovi cluster
  \item \textbf{lettura dei centroidi dei cluster} fornendo al server il path del file in cui sono salvati i centroidi da recuperare 
\end{itemize}
Il server salverà informazioni relativi agli errori nel file di logging mentre nell'interfaccia CLI notificherà l'utente quando la comunicazione si interrompe o se vi sono problemi con gli input dati. Il client da riga di comando permette di collegarsi ad una macchina che sta eseguendo un'istanza del server. L'utente nel menù potrà scegliere tra due opzioni: scoperta di cluster e lettura da file.

\subsection{Estensione}
\noindent La versione base del progetto è stata estesa con l'aggiunta di un'interfaccia grafica per smartphone Android che funge da client, supportata da un server creato utilizzando Spring Boot, e da un indirizzo di un server di proprietà del team di sviluppo. \\ Il client Android permette di selezionare il numero di cluster utilizzando uno spinner, pertanto l'utente non deve conoscere a priori il numero di transazioni presenti nel database. Inoltre, è offerta la possibilità di selezionare il file da cui leggere i centroidi con un comodo menù che permette di cercare il file desiderato dall'elenco dei file memorizzati nel server a cui si è connessi, facilitando la fruizione di tale funzione.  