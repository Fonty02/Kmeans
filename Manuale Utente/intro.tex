\section{Introduzione}

Il \textit{K-means} è un algoritmo di clustering, una tecnica di apprendimento non supervisionato utilizzata per suddividere un insieme di dati in gruppi omogenei chiamati \textit{cluster}. 
L'obiettivo del \textit{K-means} è di assegnare ogni dato al cluster più vicino, in modo che i punti all'interno di ciascun cluster siano simili tra loro e i punti tra cluster diversi siano diversi. 
Ecco come funziona l'algoritmo K-means: 
\begin{enumerate}
  \item \textbf{Inizializzazione}: Si inizia scegliendo il numero \textit{k} desiderato di cluster, e si selezionano casualmente \textit{k} punti come centroidi iniziali. Un centroide rappresenta il centro del cluster.
  \item \textbf{Assegnazione}: Per ogni punto dati, viene calcolata la distanza tra il punto e i centroidi. Il punto viene assegnato al cluster rappresentato dal centroide più vicino in base alla distanza.
  \item \textbf{Aggiornamento dei centroidi}: Una volta assegnati tutti i punti ai cluster, i centroidi vengono aggiornati calcolando la media delle posizioni dei punti all'interno di ciascun cluster. Questa media diventa il nuovo centroide per il cluster corrispondente.
  \item \textbf{Ripetizione}: I passi 2 e 3 vengono ripetuti fino a quando i centroidi smettono di cambiare o si raggiunge un numero massimo di iterazioni. In generale, l'algoritmo converge verso una soluzione stabile, anche se la soluzione ottenuta può essere un minimo locale invece del minimo globale.
  \item \textbf{Risultato}: Alla fine delle iterazioni, si ottiene un insieme di \textit{k} centroidi e i punti dati assegnati a ciascun cluster.
\end{enumerate}